\documentclass[a4paper,12pt]{article}
\usepackage{amsmath} % For 'cases' curly brace
\begin{document}
\title{Mixed Non-Bayesian and DeGroot Learning}
\author{Pedro d'Aquino\\pdaquino@umich.edu \and Dan Clark\\ddclark@umich.edu}
\date{December 11, 2012}

\maketitle

\section{Introduction}

\cite{Jadbabaie2012}
\cite{DeGroot1974}

\section{The Model}

\subsection{The Network}

(Mixed Non-Bayesian and DeGroot nodes)

\subsection{States and Signals}

We will represent the state of the world as a probability distribution:
\begin{equation}
P^*(x)=\begin{cases}
m^* \cdot x & 0 \le x \le 1 \\
0, & \text{otherwise}
\end{cases}
\end{equation}
The slope $\{ m^* \in \mathbf{R} : -2 \le m^* \le 2 \}$ will be the defining characteristic of the distribution.  We have chosen to bound $m^*$ between -2 and 2 so that the support of $P_t$  is always over $[0,1]$.  The value of the parameter $m^*$ is the ``true'' state that each node is trying to learn.

At each timestep $t$, every Non-Bayesian node $i$ will receive a signal $s_{i,t}$ which is a value drawn from $P_t$.  These observations will be used by the nodes in the network to construct a belief about $m^*$.

\subsection{Representing Belief States}

The foremost difficulty in combining the Non-Bayesian and DeGroot models is that belief of Non-Bayesian nodes is a probability distribution over a finite set of possible states of the world.

For a DeGroot node $i$, we represent the belief state of $i$ at time t as a single value 
$m_{i,t} \in \mathbf{R}$.



Therefore it is necessary to develop a conversion between the belief states of Non-Bayesian and DeGroot nodes.


\subsubsection{Belief in Non-Bayesian Nodes}

To represent a belief about the continuous slope value $m*$ for a Non-Bayesian node we discretize $m$ in the following manner.  Let $\Theta = \{\theta_0, \theta_1,...\theta_n\}, n < \infty$ be the discrete set of possible states of the world over which each Non-Bayesian node holds a distribution of belief. Each $\theta_i$ corresponds to the belief that the slope of $P^*$ is
\begin{equation} \label{eq:theta_meaning}
m^* = \frac{4i}{n - 1} - 2
\end{equation}
As an example, if we take n = 21 belief states, then we have
\begin{equation}
\nonumber
\begin{cases}
& \theta_0: m^* = -2.0 \\
& \theta_1: m^* = -1.9 \\
& ... \\
& \theta_{20}: m^* = 2.0
\end{cases}
\end{equation}
For each Non-Bayesian node $i$, the belief at time $t$ that $\theta_i$ is the true state of the world is denoted by $\mu_{i,t}(\theta_i)$.  Thus $\{ \mu_{i,t}(\theta_1), \mu_{i,t}(\theta_2), ... \mu_{i,t}(\theta_n) \}$ is a probability distribution over the set of world states for fixed $i,t$.

\subsection{Updating DeGroot Belief}
DeGroot nodes in our model do not directly receive signals; they act merely as message-passers in the network.  For the most part our DeGroot nodes behave in a fashion similarly to nodes in a standard DeGroot-style network.  That is, a DeGroot node's $i$'s belief of the true slope value $m^*$ is given at time $t+1$ as
\begin{equation} \label{eq:degroot_update}
m_{i,t+1} = \sum_{j \in N(i)} a_{ij}m_{j,t}
\end{equation}
where $N(i)$ denotes the neighbors of $i$.  $a_{ij}$ represents the level of ``trust'' that a DeGroot node $i$ has in neighbor $j$, such that

\begin{equation}
\nonumber
\sum_{j \in N(i)} a_{ij} = 1
\end{equation}
for all DeGroot nodes $i$.  Thus the DeGroot update in equation \ref{eq:degroot_update} is simply a weighted average of its neighbors beliefs in the previous timestep.

When a DeGroot node a Non-Bayesian neighbor performs an update it must obtain a value 
$m_{j,t}$ for the belief of this node.  This is calculated as

\begin{equation}
m_{j,t} = \sum_{k=1}^n\mu_{j,t}(\theta_k)(\frac{4i}{n - 1} - 2)
\end{equation}

Here we are taking a weighted average of the $m^*$ value represented by each state $\theta_i$ as defined in equation \ref{eq:theta_meaning}, where the weights are given by $j$'s level of belief in each state.

\subsection{Updating Non-Bayesian Belief}

The update of a Non-Bayesian node in our model is handled similarly to the method laid out 
in Jadbabaie et. al. % @todo[DDC] Figure out references system

% [DDC] Just putting in the equations for the moment.

\begin{equation}
m_{i,t}(s_i) = \int_\Theta \ell_i(s_i|\theta)d\mu_{i,t}(\theta) = \sum_{i=1}^n \ell_i(s_i|\theta_i)\mu_{i,t}(\theta_i)
\end{equation}

\begin{equation}
\mu_{i,t+1}(\theta_i) = a_{ii}\mu_{i,t}(\theta_i)\frac{\ell_i(\omega_{i,t+1}|\theta_i)}{m_{i,t}(\omega_{i,t+1})} + \sum_{j \in N(i)} a_{ij}\mu_{j,t}(\theta_i)
\end{equation}



\bibliographystyle{plain}
\bibliography{eecs598Report}

\end{document}
